\begin{resumo}[Abstract]
 \begin{otherlanguage*}{english}
   
Test data generation is a current challenge on tests area in software engineering. Develop the exact tests is not always a simple task to do. A strategy that is being used is to treat data generation as an optimization problem and apply methods of approximation, such as metaheuristics. An identified problem is that data generation process tend to be single objective, analysing only the code coverage. However, often the data generation and the coverage execution required a significant computational effort. In addition, there are few studies that aim at a performance analysis of metaheuristics. The research objetive is to search the otimization point of the relations between the coverage and the computational effort used in the test data generation and the coverage and computational effort in the execution of the tests. For this, the behavior of the Genetic Algorithm, metaheuristic enough employed on test data generation, will be analyzed under the presented objectives. As a consequence, we intend to observe what stands out and perform a particular investigation in order to establish parameters that can increment this metaheuristic type for test data generation.

   \vspace{\onelineskip}
 
   \noindent 
   \textbf{Key-words}: optimization problem. Genetic Algorithm. test data generation. code coverage. computational effort
 \end{otherlanguage*}
\end{resumo}