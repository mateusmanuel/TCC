\begin{resumo}[Abstract]
 \begin{otherlanguage*}{english}
   
Test data generation is a current challenge in Software Engineering and develop an accurate set of data for testing is not a simple task to do. A strategy that is being used is to treat test data generation as an optimization problem and apply approximation methods, also called metaheuristics. An identified problem is that data generation process tend to be single objective, analysing only the code coverage. However, it is not a realistic scenario since other objectives can (and should) be considered when proposing a set of tests. As an example, there seems to be a balance between computational efforts to generate test data and the execution of it coverage. In addition, studies are still needed to analyze the performance of metaheuristics used in the generation / execution of tests. The research objetive is to investigate and analyze the relationships a) between the code coverage and the computational effort employed in test data generation and b) the coverage and computational effort in test execution. For this, the graphical behavior of the genetic algorithm was analyzed, metaheuristic used in the generation of test data. As a consequence, we intend to observe what stands out and perform a particular investigation in order to establish parameters that can increment this metaheuristic type for test data generation.

   \vspace{\onelineskip}
 
   \noindent 
   \textbf{Key-words}: optimization problem. Genetic Algorithm. test data generation. code coverage. computational effort
 \end{otherlanguage*}
\end{resumo}