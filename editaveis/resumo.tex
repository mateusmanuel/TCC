\begin{resumo}
 
Geração de dados de teste é um desafio corrente da área de testes na engenharia de software. Desenvolver os testes exatos nem sempre é uma tarefa simples de ser realizada. Uma estratégia que vem sido empregada é tratar a geração de dados como um problema de otimização e com isso aplicar métodos de aproximação, as meta-heurísticas. Um problema identificado no processo de geração de dados é que eles tem objetivo simples, analisam apenas o cobertura de código que está sendo alcançado. Porém, muitas vezes, a geração dos dados e a execução da cobertura requerem um esforço computacional significativo. Além disso, se carece de estudos que visam a análise de desempenho das meta-heurísticas. Dessa forma, o objetivo da pesquisa é buscar o ponto de otimização das relações entre a cobertura e o esforço computacional empregado na geração de dados de teste e a cobertura e esforço computacional na execução dos testes. Para isso, será analisado o comportamento gráfico do algoritmo genético, meta-heurística bastante empregada na geração de dados de teste, sob o olhar dos objetivos apresentados. Como consequência, pretende-se observar qual o esforço que se destaca e realizar uma investigação em especial a fim de estabelecer parâmetros que possam a incrementar esse tipo de meta-heurística para a fim de geração de dados de teste.

 \vspace{\onelineskip}
    
 \noindent
 \textbf{Palavras-chaves}: problema de otimização. algoritmo genético. geração de dados de teste. cobertura de código. esforço computacional
\end{resumo}
