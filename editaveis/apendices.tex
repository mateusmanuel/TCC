\begin{apendicesenv}

\partapendices

\chapter{Scripts utilizados no estudo}
\label{scripts}

\begin{listing}
    \begin{minted}[
        frame=lines,
        framesep=2mm,
        baselinestretch=1.2,
        fontsize=\footnotesize,
        linenos
        ]{bash}
    #/bin/bash
    
    TARGET_CLASS=$1
    PATH_PROJECT=$2
    SUBPATH_PROJECT=$3
    TIME=1
    
    for TIME in `seq 2 2 20`
    do
        cd $PATH_PROJECT
        mvn clean compile
    
        rm -r evosuite-tests
    
        $EVOSUITE -class $TARGET_CLASS -projectCP target/classes -Dsearch_budget=$TIME 
    -Dalgorithm=STANDARD_GA -Doutput_variables=search_budget,TARGET_CLASS,Coverage,Total_Time
    
        rm -r src/test  
        mkdir -p src/test/java/$SUBPATH_PROJECT
        cp evosuite-tests/$SUBPATH_PROJECT/*.java src/test/java/$SUBPATH_PROJECT
    
        mvn clover:setup test clover:aggregate clover:save-history
    done
    \end{minted}  
    \caption{\textit{run.sh}. Script de execução para geração do testes e execução da cobertura de código}  
    \label{run.sh}
\end{listing}


\begin{listing}
    \begin{minted}[
        frame=lines,
        framesep=2mm,
        baselinestretch=1.2,
        fontsize=\footnotesize,
        linenos
        ]{bash}
    #/bin/bash

    for FILE in *.xml.gz; do
        gunzip $FILE
    done

    echo "seconds,coverage,time" > statistics-execute.csv

    I=1
    for FILE in *.xml; do
        sed -n '2,4p;5q' $FILE > $FILE.out

        # Calculate delta time
        END=$(grep -oP "(?<=generated=\")[^ ]+" $FILE.out)
        END=${END%?}
        
        START=$(grep -oP "(?<=timestamp=\")[^ ]+" $FILE.out)
        START=${START%?}
        START=${START%?}

        TOTAL_TIME=`expr $END - $START`
        echo $TOTAL_TIME

        # Calculate coverage
        COV_ELEMENTS=$(grep -oP "(?<=coveredelements=\")[^ ]+" $FILE.out)
        COV_ELEMENTS=${COV_ELEMENTS%?}

        UNCOV_ELEMENTS=$(grep -oP "(?<= elements=\")[^ ]+" $FILE.out)
        UNCOV_ELEMENTS=${UNCOV_ELEMENTS%?}
        echo $UNCOV_ELEMENTS

        COVERAGE=$(bc -l <<< "scale=16; $COV_ELEMENTS/$UNCOV_ELEMENTS")

        echo "$I,$COVERAGE,$TOTAL_TIME" >> statistics-execute.csv
        ((I++))
    done
    \end{minted}
    \caption{\textit{extract.sh}. Script de extração das métricas dos relatórios de execução da cobertura de código}  
    \label{extract.sh}
\end{listing}


\begin{listing}
    \begin{minted}[
        frame=lines,
        framesep=2mm,
        baselinestretch=1.2,
        fontsize=\footnotesize,
        linenos
        ]{python}
        #!/usr/bin/python

        import matplotlib.pyplot as plt
        import pandas as pd
        import numpy as np
        import sys

        df = pd.read_csv('statistics_' + sys.argv[2] + '.csv')
        df.loc[:,'coverage'] *= 100.0

        font = {'family' : 'serif',
                'size' : 10 }
        plt.rc('font', **font)

        fig, ax1 = plt.subplots()

        x = np.arange(2, 21, step=2)
        ax1.plot(x, df['coverage'], 'C0-', linewidth=2.0, label="test1")
        ax1.set_xlabel('Massa de dados')

        ax1.set_ylabel('Cobertura (%)', color='C0')
        ax1.tick_params('y', colors='C0')
        ax1.xaxis.set_tick_params(labelsize=7)

        ax2 = ax1.twinx()
        ax2.plot(x, df['time'], 'C1--', linewidth=2.0, label="test2")
        ax2.set_ylabel('Esforço computacional (ms)', color='C1')
        ax2.tick_params('y', colors='C1')

        plt.xticks(x)

        fig.tight_layout()
        fig = plt.gcf()
        fig.savefig(sys.argv[2] + '_' + sys.argv[1] + '.png')

        plt.show()
        \end{minted}
        \caption{\textit{plot.py}. Script para plotagem dos gráficos}  
        \label{plot.py}
\end{listing}


\end{apendicesenv}
