\chapter[Considerações Finais]{Considerações Finais}
\label{conclusoes}

Na presente pesquisa foi abordado um conceito cada vez mais difundido, que é a otimização de problemas em que a solução exata demanda um tempo grande e até mesmo não-determinístico. Além disso, procurou discorrer sobre as meta-heurísticas, métodos de solução usualmente empregados na resolução desse tipo de problema, avaliando principalmente as que são mais recorrentes na Engenharia de Software Baseada em Busca. Aprofundando na área foi possível observar a importância dos testes nessa linha de pesquisa principalmente os testes estruturais. A geração de dados de teste se apresentou como a primeira técnica a empregar meta-heurísticas e até hoje é bastante utilizada. Dentre as técnicas abordadas os algoritmos genéticos se apresentaram como uma oportunidade de se estudar o seu comportamento.

A observância de se estudar aspectos não-funcionais dos testes contribui para a relevância da pesquisa. A análise desse tipo de parâmetro, como o esforço computacional empregado, se faz cada mais necessária na área principalmente para que se consiga o aperfeiçoamento das técnicas empregadas. Na maioria dos casos, a abordagem em multiobjetivos é o que faz os testes produzidos serem mais realistas. 

Antes mesmo da realização e conclusão da segunda parte da pesquisa, é possível inferir que esse tipo de estudo é importante não somente para testes, mas também para as outras áreas da engenharia de software. Ainda mais porque são estudos emergentes, podendo contribuir em muito no desenvolvimento das mesmas. A expansão para outras meta-heurísticas também é algo a ser pensado, pois cada uma tem suas próprias características e peculiaridades a serem examinadas e relacionadas a outros fatores.

(REVISAR ESSE CAPÍTULO AO FINAL DO TCC)
