\chapter[Introdução]{Introdução}

A busca pela ``melhor'' configuração ou conjunto de parâmetros para a resposta de um determinado objetivo/problema é uma preocupação recorrente de problemas práticos e teóricos \cite{combinatorialoptimization1998}. Em muitos desses problemas, realizar uma busca exaustiva por todos os estados do espaço de solução (\textit{i.e.} todas as combinações de soluções possíveis) a fim de encontrar a melhor solução ou uma condição ideal não é uma alternativa viável do ponto de vista prático, devido a enorme quantidade de possibilidades a serem testadas. 

A Ciência da Computação também lida comumente com esses tipos de problemas desde seus primórdios. \citeonline{menabrea1842sketch} relatam tais problemas em suas notas~\cite{menabrea1842sketch}. Um caso particular está descrito na nota D, em que os autores expõem que é preferível escolher um arranjo que tenderá a reduzir o tempo necessário para completar a realização de um cálculo, na máquina de calcular de Charles Babbage, do que necessariamente encontrar o arranjo exato da melhor solução possível~\cite{menabrea1842sketch}.

A otimização de sistemas é um processo de melhoramento de soluções iterativo e até mesmo interativo, tendo em vista sempre uma função objetivo. Em linhas gerais, os métodos de otimização formulam um problema e suas restrições através de uma ou várias funções matemáticas (\textit{a.k.a.} função objetivo) e buscam encontrar em tempo hábil uma resposta descrita por um conjunto de parâmetros.  Tal processo de resolução de sistemas apresenta-se como uma possibilidade de solução bastante adotada para problemas cujo espaço de solução oferece inúmeras possibilidades \cite{snyman2005practical}. 

Um tipo especial desses problemas é chamado de problema multi-objetivos, sendo caracterizado por uma função matemática representando diversos objetivos a serem atingidos tendo eles, muitas vezes, propósitos diferentes. Um exemplo a se citar é a busca por uma solução que a) maximize os ganhos de um sistema com b) o menor esforço possível.  Para esses problemas os métodos de aproximação, também chamados de heurísticas, geram resultados de alta qualidade em um tempo razoável para uso prático e têm sido amplamente empregados como técnicas para otimizar a solução de problemas~\cite{gendreau2005metaheuristics}. Com o desenvolvimento dessa área de pesquisa os métodos de aproximação foram sendo melhor estruturados e conceituados e assim definiram um nível mais generalista e consolidado de heurísticas, as chamadas meta-heurísticas~\cite{talbi2009metaheuristics}.

Na Engenharia de Software o primeiro trabalho a empregar meta-heurísticas em um problema de otimização foi na área de testes, especificamente na geração de dados de teste \cite{miller1976automatic}. Em 2001, em um manifesto elaborado por \citeonline{harman2001search} foi cunhado o termo Engenharia de Software Baseado em Busca (do inglês \textit{Search-Based Software Engineering - SBSE}) no qual os autores propõem a reformulação da Engenharia de Software como um problema de busca~\cite{harman2001search}. A estruturação proposta pelos autores provocou a evolução de outras áreas na Engenharia de Software tais como a estimativa de custos e o gerenciamento de projetos. Todavia, os testes continuam dominando o interesse da comunidade de modo que foi estabelecida  uma área própria denominada de Testes de Software Baseados em Busca - em uma tradução livre do inglês \textit{Search-Based Software Testing - SBST}~\cite{harman2012search}.

A geração de dados para teste como uma maneira de substituir a escolha empírica dos dados realizada pelo testador destaca-se como um dos principais tópicos pesquisados em SBST~\cite{mcminn2004search}. A grande maioria das abordagens de geração de dados a partir de meta-heurísticas são de objetivo único estando elas preocupadas em aumentar a cobertura do código que está sendo testado~\cite{harman2015achievements}. Entretanto, na realidade, é desejável que os testes multi-objetivos, que consigam atingir outros objetivos além de determinados níveis de cobertura. Por exemplo, produzir um teste que busca aumentar a cobertura de código ao mesmo tempo em que se mantem exequível, respondendo ao desenvolvedor ou testador dentro de um intervalo de tempo adequado. Há, portanto, uma série de características de testes de software que são de interesse dos desenvolvedores e testadores mas que geralmente não são consideradas de modo conjunto~\cite{harman2015achievements}.

Dentre as características de testes identificadas na literatura duas merecem especial atenção quanto ao esforço computacional demandado. Por um lado o esforço para a geração de dados de teste parece ter uma relação direta com o nível de cobertura atingido. Intuitivamente, quanto maior o nível de cobertura desejado, provavelmente mais casos de testes serão necessários e, portanto, mais dados de testes deverão ser gerados. Por outro lado, a execução de testes tem, certamente, uma relação direta com o esforço empreendido em tal tarefa de modo que quanto mais testes a serem executados maior será o esforço computacional empreendido.  Se ambas características forem consideradas em conjunto, tem-se como objetivos de um conjunto de testes maximizar a cobertura de código  ao mesmo tempo em que busca-se gerá-los e executá-los no menor tempo possível.

Isso exposto nota-se a necessidade de investigar outras características desejáveis em testes de software e o modo como elas estão relacionadas. No contexto desse trabalho a investigação se restringe à relação entre os esforços de geração de dados de teste e sua execução, o que estabelece a seguinte questão de pesquisa:

 
\vspace{1.0cm}
\fbox{
\parbox{0.85\textwidth}{\textbf{Questão de pesquisa:} Como o esforço de geração de dados de teste e de execução de testes se relacionam? }}
\vspace{1.0cm}


Portanto, o objetivo deste trabalho é compreender o ponto de equilíbrio nas relações existentes na geração/execução de dados de teste a saber: i) o esforço computacional empregado na geração dos dados de teste e a cobertura de código provida por tais dados e ii) o esforço computacional para a execução dos testes e o nível de cobertura de código alcançado. Em outras palavras, estima-se que o ponto de equilíbrio entre essas duas relações caracteriza  uma solução de testes com boa cobertura de código e tempo de execução factível. Para tal estudo ambas características serão aplicadas à abordagens que utilizam algoritmos genéticos devido à sua larga utilização nessa área e outros fatores, como alta adaptabilidade e especifidade a problemas de otimização, que estão descritos em detalhes na Seção \ref{propostaestudo}. Além disso, a compreensão do comportamento das iterações do algoritmo genético é importante na identificação da sua interferência na cobertura produzida. Com essas definições, será possível descrever as funções de relação e estabelecer a otimização das mesmas. 


O trabalho está estruturado em quatro principais capítulos. O Capítulo~\ref{referencialteorico} apresenta o referencial teórico da pesquisa em que são abordados os conceitos e classificações inerentes à otimização; as meta-heurísticas como métodos probabilísticos de solução para o problemas de solução, caracterizando as principais meta-heurísticas utilizadas na Engenharia de Software; a geração de dados de teste baseado em busca, expressando conceitos como cobertura de código e testes multi-objetivos; e o esforço computacional de teste. No Capítulo~\ref{propostaestudo} é desenvolvida a proposta de estudo, mostrando a justificativa e os objetivos da pesquisa. No Capítulo~\ref{estudoexploratorio} é mostrado o desenvolvimento do estudo, bem como os resultados e as discussões levantadas. O Capítulo~\ref{conclusoes}, conclui as ideias apresentadas e mostra as expectativas para o seguimento da pesquisa.