\chapter[Proposta de Estudo]{Proposta de Estudo}
\label{propostaestudo}

A produção de dados para testes de forma automatizada é um dos principais avanços na área de Engenharia de Software \cite{harman2015achievements}. Isso ocorre devido à necessidade de reproduzir comportamentos para a verificação de um código ser fundamental para agregar valor ao software. A geração desses dados por meta-heurísticas complementam e contribuem com as atividades típicas de um desenvolvedor. Todavia, nota-se que nem sempre esses testes gerados são realísticos. Em alguns casos acaba sendo custoso tanto o processo de geração dos dados quanto suas execuções para medir a cobertura de código.  Tais fatores se tornam, portanto, empecilhos para o crescimento e adoção das meta-heurísticas \cite{harman2015achievements}.

Outro fator observado é que, em geral, há esforços empreendidos em produzir e evoluir as meta-heurísticas, mas pouco se estuda e analisa essas técnicas aplicadas em um determinado cenário. A geração de dados de teste carece de estudos na avaliação do comportamento dos algoritmos empregados para a otimização de acordo com os testes produzidos. Algumas métricas que podem ser analisadas são, por exemplo, o tempo de execução e o número de iterações desses algoritmos \cite{rodrigues2018using}. 

Os algoritmos genéticos tem sido empregados com bastante frequência para a geração de dados para teste \cite{rodrigues2018using}. Isso porque a geração de dados de teste é formulado como um problema de busca em que a solução consiste em encontrar os dados que ao serem testados vão atingir o objetivo do teste, que como mostrado em maioria é a cobertura. A estruturação de tais algoritmos em passos (recombinação - mutação - avaliação - seleção) garante uma abrangência e ao mesmo tempo uma especificidade para os dados de teste. Os dois primeiros passos são fundamentais para promoção da variabilidade genética enquanto que os últimos dois, tomando como base a função objetivo da otimização, restringem bem as gerações \cite{rodrigues2018using}. Esses passos são passíveis de otimização e por isso podem ser analisados a fim de que se compreenda o impacto na geração e na execução dos testes.

Dadas tais justificativas, abaixo são apresentados os dois principais objetivos da pesquisa, seguido do objetivo geral a ser alcançado. 

\makeatletter
\def\namedlabel#1#2{\begingroup
    #2%
    \def\@currentlabel{#2}%
    \phantomsection\label{#1}\endgroup
}
\begin{description}
\item[\namedlabel{obj1}{Objetivo 1}] Analisar a relação entre cobertura do código e o esforço computacional para geração de dados de teste utilizando algoritmo genético.

A geração de dados para teste está ligada aos processos iterativos da meta-heurística empregada. Os algoritmos genéticos têm a característica de quanto maior sua geração mais custoso é o seu processo de evolução \cite{pargas1999test}. Por outro lado, essas gerações tendem a proporcionar dados de teste mais específicos. Intuitivamente, percebe-se que é uma relação crescente mas não se sabe, a princípio, sua forma (linear, logarítmica, exponencial, dentre outras). Portanto o objetivo é compreender melhor o comportamento conjunto dessas duas variáveis a saber, explicitamente, esforço computacional de geração de dados de teste e o  nível de cobertura de teste atingido. 
		

\item[\namedlabel{obj2}{Objetivo 2}] Analisar o esforço computacional empregado para execução da cobertura de código a partir dos testes gerados por algoritmo genético.

A depender de como os dados de teste são gerados eles podem provocar um determinado grau de esforço computacional a ser exigido na cobertura de código.  Nesse aspecto, também é possível analisar que quanto mais testes são executados, mais esforço computacional é necessário e, possivelmente, maior será a cobertura alcançada. Intuitivamente, percebe-se novamente que é uma relação crescente mas também não se sabe a forma da função matemática que descreve a relação entre testes gerados por algoritmos genéticos e o esforço computacional de sua execução. Portanto o que se deseja é entender o modo que essas variáveis se relacionam para os algoritmos genéticos.

\end{description}

Com base nesses objetivos o que se deseja é encontrar o ponto de equilíbrio dessas relações, ou seja,  o ponto ótimo para ambas análises. Sendo assim,  o estudo dessa pesquisa pode-se caracterizar como um problema de otimização no qual se deseja diminuir o esforço computacional empregado para a geração de dados e execução dos teste e maximizar a cobertura de código. Como mostrado, este é um problema multi-objetivo que se enquadra na área de Testes de Software Baseado em Busca.

Vale ressaltar que esforço computacional na proposta de estudo está ligado ao esforço do teste, no caso, o tempo de execução da duas atividades mencionadas nos objetivos: a geração de testes e a execução de testes. Não está relacionado com o complexidade do algoritmo com relação ao tempo. Por isso, no estudo não se levantará e estudará o tempo com base na análise do algoritmo e se o tempo de execução.


% (REMOVER TODA ESSA SEÇAO NA PROXIMA VERSAO)

% \section{Cronograma}

% De acordo com a proposta que foi apresentada algumas atividades foram planejadas para dar seguimento na pesquisa.

% \begin{description}
% \item[\namedlabel{ativ1}{Atividade 1}] Estudar o comportamento das relações referentes ao \ref{obj1} e \ref{obj2}.

% O objetivo dessa atividade é conseguir analisar as variáveis que influenciam a geração e execução dos testes. E necessário delimitar as propriedades da meta-heurística escolhida. Com isso, será possível pegar exemplos de uso e medi-las, a fim de identificar pontos fora da normalidade. Nesse ponto, os objetivos específicos serão observados separadamente.

% \item[\namedlabel{ativ2}{Atividade 2}] Estudar e definir o modo como os esforços computacionais para o \ref{obj1} e \ref{obj2} podem ser medidos.

% A partir do momento em que as pontos divergente do comum forem surgindo na atividade anterior, será possível começar a análise do esforço computacional empregado. A identificação de padrões desses esforços é um ponto a ser bastante examinado. Como resultado é esperado a definição da métrica de esforço computacional a ser provada.

% \item[\namedlabel{ativ3}{Atividade 3}] Propor e analisar uma relação entre o \ref{obj1} e \ref{obj2} que capture os esforços de ambos.

% Essa atividade tem como propósito entender o comportamentos das duas curvas das relações se equiparam, ou seja, buscar a otimização desse problema. Para tanto, será definido a função objetivo, as variáveis e as restrições do otimização. 

% \item[\namedlabel{ativ4}{Atividade 4}] Escrita dos resultados e discussões

% A representação gráfica e sistemática das outras atividades é fundamental para a validade da pesquisa. Assim, essa atividade tem como intuito estruturar e consolidar as informações recolhidas e geradas durante o estudo. Apresentar o debate sobre as questões, os desafios de validação e a expansão para trabalhos futuros também está dentro dessa atividade.

% \end{description}

% A ordem das atividades obedece a dependência entre elas e a prioridade de cada uma. A que demanda mais tempo e que representa o objetivo geral da pesquisa é a \ref{ativ3}, por isso foi separado três meses para a sua realização. A tabela \ref{cronograma} estabelece o cronograma para a próxima parte da pesquisa, e está divida em meses na vertical e as atividades na horizontal. 

% \begin{table}[!htbp]
% \centering
% \caption{Cronograma das atividades durante os meses de agosto à dezembro}
% \label{cronograma}
% \begin{tabular}{|l|c|c|c|c|c|}
% \hline
% Mês & \multicolumn{1}{l|}{\multirow{2}{*}{Agosto}} & \multicolumn{1}{l|}{\multirow{2}{*}{Setembro}} & \multicolumn{1}{l|}{\multirow{2}{*}{Outubro}} & \multicolumn{1}{l|}{\multirow{2}{*}{Novembro}} & \multicolumn{1}{l|}{\multirow{2}{*}{Dezembro}} \\ \cline{1-1}
% Atividade & \multicolumn{1}{l|}{} & \multicolumn{1}{l|}{} & \multicolumn{1}{l|}{} & \multicolumn{1}{l|}{} & \multicolumn{1}{l|}{} \\ \hline
% \ref{ativ1} & X & X &  &  &  \\ \hline
% \ref{ativ2} &  & X & X &  &  \\ \hline
% \ref{ativ3} &  & X & X & X &  \\ \hline
% \ref{ativ4} &  &  &  & X & X \\ \hline
% \end{tabular}
% \end{table}










