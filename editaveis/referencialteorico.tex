%\part{Referencial Teórico}

\chapter[Referencial Teórico]{Referencial Teórico}
\label{referencialteorico}

Esse capítulo apresenta, em linhas gerais, os elementos necessários para o entendimento da proposta desse trabalho. Inicialmente será apresentado o que são problemas de otimização e como eles se caracterizam (Seção \ref{sec:probOtimizacao}) seguida pela definição e exemplificação de meta-heurísticas (Seção \ref{sec:metaheuristica}). Por fim, os conceitos relacionados à geração de dados de teste sob a perspectiva de SBST serão apresentados na Seção \ref{sec:trabsSBST}.

\section{Problema de otimização \label{sec:probOtimizacao}}

A otimização é um termo que vem da matemática, a qual em sua essência está preocupada com a obtenção das condições que dão o valor extremo de uma função, ou de várias funções, sob determinadas circunstâncias. A definição formal matemática pode ser dada como: 

\begin{eqnarray}
\label{eqn01}
	 \textsf{Encontrar \textbf{x} que minimize \textbf{f(x)} sujeito a} \nonumber \\
     g_j(x) \leq 0, \qquad j = 1, 2, ..., n_g \\
\label{eqn02}
      h_k(x) = 0, \qquad k = 1, 2, ..., n_h \\
\label{eqn03}
      x_{i}^{L} \leq x_i \leq x_{i}^{U}, \qquad i = 1, 2, ..., n 
\end{eqnarray}

onde $x_{e}$ o vetor das n variáveis de projeto, $f(x)$ é a função objetivo e $g_{j}(x)$ e $h_{k}(x)$ são as restrições de desigualdade e igualdade, respectivamente. Os limites das variáveis são determinados através da equação \ref{eqn03}, onde onde $x_{i}^{L}$ é o limite inferior e $x_{i}^{U}$ é o limite superior da variável $x_{m}$. Nas outras expressões, $n$, $n_{g}$ e $n_{h}$ são o número de variáveis de projeto, número de restrições de desigualdade e igualdade, respectivamente \cite{gandomi2013metaheuristic}.

Na definição, a função objetivo se refere ao que se quer maximizar ou minimizar, o conjunto de variáveis é o conjunto de valores que afeta o valor da função objetivo e o conjunto de restrições é o conjunto que não permite que o conjunto de variáveis assuma determinados valores. Considerando tais elementos é possível classificar os problemas de otimização de acordo com as características de cada um deles. Os possíveis valores que cada elemento pode assumir em tal classificação estão descritos na Figura \ref{figotimizacao01}.

Os problemas típicos de otimização vão desde a minimização de funções algébricas até assuntos mais práticos em projetos de engenharia tais como a minimização de custos e a maximização de desempenho. A pergunta que todos esses problemas têm em comum é ``Dentre as soluções viáveis, qual é a melhor?''. O meio que se busca para respondê-la, após a definição das restrições e da função objetivo, é pela utilização de algum método de otimização. Os métodos de otimização que se ajustam a pergunta são os probabilísticos, no caso as heurísticas e meta-heurísticas~\cite{gandomi2013metaheuristic}.

\begin{figure}[h]
	\centering
	\label{figotimizacao01}
    \tikzset{
        basic/.style  = {draw, text width=4cm, drop shadow, font=\sffamily, rectangle},
        root/.style   = {basic, rounded corners=2pt, thin, align=center,
                         fill=gray!60},
        onode/.style = {basic, thin, align=center, fill=gray!40,text width=4cm,},
        tnode/.style = {basic, thin, align=left, fill=gray!20, text width=6.5em},
        edge from parent/.style={-, >={latex}, draw=black, edge from parent fork right}
    }

    \begin{tikzpicture} [%
        grow=right,
        anchor=west,
        growth parent anchor=east,
        parent anchor=east,
        level 1/.style={sibling distance=6em},
        level 2/.style={sibling distance=2.5em},
        level distance=0.5cm]
    \node[root] (root) {Classificação dos problemas de otimização}
        child {node[onode] (c1) {Número de variáveis de projeto}
            child {node[tnode] (c11) {Única Variável}}
            child {node[tnode] (c12) {Multivariável}}
        }
        child {node[onode] (c2) {Número de funções objetivo}
            child {node[tnode] (c21) {Único Objetivo}}
            child {node[tnode] (c22) {Multiobjetivo}}
        }
        child {node[onode] (c3) {Presença de restrições}
            child {node[tnode] (c31) {Sem restrições}}
            child {node[tnode] (c32) {Com restrições}}
        }
        child {node[onode] (c4) {Tipo de variáveis de projeto}
            child {node[tnode] (c41) {Discreta}}
            child {node[tnode] (c42) {Contínua}}
            child {node[tnode] (c43) {Mista}}
        }
        child {node[onode] (c5) {Característica das restrições e da função objetivo}
            child {node[tnode] (c51) {Linear}}
            child {node[tnode] (c52) {Não-linear}}
        }
        child {node[onode] (c5) {Natureza das variáveis e dados de entrada}
            child {node[tnode] (c51) {Determinística}}
            child {node[tnode] (c52) {Probabilística}}
        };
    \end{tikzpicture}
    \caption{Classificação dos problemas de otimização \cite{gandomi2013metaheuristic}}
\end{figure}

\section{Meta-heurísticas \label{sec:metaheuristica}}

Os métodos de busca meta-heurísticos podem ser definidos como metodologias ou 
modelos de nível superior que podem ser usados como estratégias de orientação
no projeto de heurísticas subjacentes para resolver problemas de
otimização \cite{talbi2009metaheuristics}. Ser de nível superior significa que são algoritmos de propósito geral, enquanto as heurísticas são desenhadas e construídas para  problemas específicos.

Como dito, elas costumam ser usadas como estratégia de orientação para o projeto de heurísticas derivadas. Isso expressa que normalmente elas são adaptadas de acordo com as restrições, as variáveis e a função objetivo do problema em questão.

Em uma meta-heurística são observados dois principais fatores: a investigação do espaço de busca (diversificação) e a exploração das melhores soluções encontradas (intensificação). Esses fatores, geralmente estão ligados ao tipo de solução em que estão baseadas as meta-heurísticas. Aquelas meta-heurísticas baseadas em solução única, tais como o recozimento simulado e a busca local, estão ligadas a intensificação ao passo que as meta-heurísticas baseadas na população, como algoritmos genético e a inteligência de enxame, estão relacionadas a diversificação \cite{talbi2009metaheuristics}.

As meta-heurísticas possuem uma grande variabilidade de classificação, que vão além dos fatores mostrados anteriormente. Um outro que pode ser citado é a inspiração, podendo ser baseada em processos da natureza (a grande maioria dos casos) e não-baseada em processos naturais. A maneira de decisão dos algoritmos classifica as meta-heurísticas em determinísticas e estocásticas em que regras aleatórias são aplicadas durante a busca. Dado essa diversidade as meta-heurísticas Subida de Encosto, Recozimento Simulado,  Algoritmo Genético e  Inteligência de Enxame ganham destaque, principalmente na área de Engenharia de Software \cite{khari2017extensive} por tratarem, principalmente, de problemas estocásticos.

\subsection{Subida de Encosta}

O algoritmo da Subida de Encosta (\textit{Hill Climbing}) é um método simples que proporciona resultados rápidos, e por isso é um dos algoritmos mais utilizados. Ele é baseado no algoritmo amplamente conhecido e utilizado para travessia em grafos para buscas em profundidade.

A subida de encosta tem como ponto de partida o espaço de busca e um resultado prévio escolhido arbitrariamente. A cada iteração, os vizinhos desse resultado são examinados. A solução atual é substituída assim que se encontra uma solução melhor. Isso ocorre sucessivamente até que se não consiga encontra vizinhos com melhores resultados.  

\subsection{Recozimento Simulado}

Recozimento Simulado (\textit{Simulated Annealing}) é um algoritmo similar a Subida de Encosta com a diferença que os movimentos no espaço de busca não são tão limitados. Para a exploração do espaço de busca ele utiliza um parâmetro de controle denominado temperatura. Tal parâmetro é a probabilidade de aceitar as piores soluções, ou seja, soluções com menor valor de aptidão. Aceitar as piores soluções é uma propriedade fundamental porque possibilita uma busca mais extensiva para a solução ótima global \cite{kirkpatrick1983optimization}. 

O método consiste em iniciar com um valor alto de temperatura e à medida em que é aplicado o resfriamento, ou seja, a busca, a temperatura tende a diminuir até chegar a zero. Tal como a Subida de Encosto, o Recozimento Simulado considera apenas uma solução por iteração e não realiza qualquer suposição sobre o panorama de soluções. Outro problema que possui igualmente à Subida de Encosto é a possibilidade de ficar preso em um ótimo local quando a temperatura esfria rapidamente.

\subsection{Algoritmo genético}

Os Algoritmos Genéticos (\textit{Genetic Algorithms} - GA) partem semelhança da codificação dos resultados de candidatos como uma série de componentes simples e o arranjo genético de um cromossomo \cite{alander1998genetic}. Os resultados, nesse caso, geralmente se referem aos indivíduos (cromossomos) que utilizam essa meta-heurística. Os possíveis valores para cada componente é chamado de alelo, a sua posição específica na sequência é denominada locus e os constituintes do resultado são por vezes denotados como genes.

A escolha de uma população inicial aleatória, conjunto de cromossomos, é o primeiro passo do processo iterativo dos algoritmos genéticos. As iterações, denominadas de gerações, terminam quando a condição pré-determinada é atingida ou quando o número de gerações é excessivo. Em cada geração, cromossomos são recombinados cruzando seus genes. Uma parte da descendência dessa união sofre mutação e, a partir da prole e da população original, um processo de seleção é usado para determinar a nova população. Crucialmente, a recombinação e a seleção são guiadas pela função objetivo; cromossomos mais aptos tendo uma chance maior de serem selecionados e recombinados.

\subsection{Inteligência de Enxame}

A meta-heurística Inteligência de Enxame (\textit{Swarm Intelligence}) tem como base o modelo biológico, com ênfase em como os indivíduos trabalham em conjunto com a distribuição de informações. As redes de fluxos de feromônios são o principal objetivo das formigas para decidir onde forragear. Se as formigas encontrarem aleatoriamente um obstáculo, elas procurarão métodos ao redor. No entanto, quando certas formigas encontram uma maneira de contorná-lo, as outras formigas seguem sua pista de feromônio para criar uma nova rota.

A cooperação das formigas foi a primeira forma de organização biológica analisada. Porém, com o passar outros métodos de otimização de inteligência de enxame surgiram, como a Otimização por Enxame de Partículas (PSO), Otimização por Colônia de Abelhas (ABC) e a Otimização por Sistemas Artificiais Imunológicos (AIS) \cite{blum2008swarm}. Em todos esses são observados determinados comportamentos da colônia como um todo, e de cada indivíduo, criando um inteligência artificial própria, que vão agregar e constituir a formulação da função objetivo do problema de otimização. As iterações vão obedecer as restrições biológicas do enxame escolhido.

\section{Geração de Dados de Teste de Software Baseada em Busca \label{sec:trabsSBST}}

% As definições citadas acimas contribuem para o desenvolvimento da engenharia de software. Com tais, é possível observar a generalidade que se ganha com a representação de um problema e com a função objetivo, e a robustez que os algoritmos de otimização proporcionam \cite{harman2012search}. 

% Além disso, pode ser observada a escalabilidade através do paralelismo que as meta-heurísticas proporcionam, a reunificação de subáreas da engenharia de software e a computação direta dos problemas \cite{harman2012search}. Isso, devido ao software já estar representada de maneira lógica, diferente de outras engenharias, que precisam da simulação dos seus artefatos. Como por exemplo, a representação de um material físico, o que necessita de um alto poder computacional.

A otimização por busca se encontra difundida em testes, pois é possível notar a atuação em todas as dimensões do teste. Seja ele estrutural (caixa branca), funcional (caixa-preta), uma mescla de ambos (caixa cinza) e não-funcional. Entretanto, os testes estruturais foram os que mais receberam esforços para o desenvolvimento. Principalmente, visando a geração de casos e dados de teste para a cobertura de caminhos possíveis na execução \cite{khari2017extensive}. 

A geração de dados para teste constituti na busca de valores de entrada que atendam a critérios de teste específicos. Estes valores de entrada podem assumir diferentes formas, tais como parâmetros de uma função, parâmetros para uma sequência de chamadas de método, dentre outras. A finalidade que mias se almeja com a geração de dados é cobertura de código de casos que demandariam um empenho grande do testador.

\subsection{Cobertura de Código}

A cobertura de teste é uma medida usada para descrever o grau em que o código-fonte de um programa é executado quando um determinado conjunto de testes é aplicado. Um programa com alta cobertura de teste, medido em porcentagem, tem o seu código-fonte mais executado durante o teste, o que sugere que ele tem uma chance menor de conter erros de software não detectados em comparação com um programa com baixa cobertura de teste \cite{yang2009survey}. Muitas métricas diferentes podem ser usadas para calcular a cobertura de teste; algumas das mais básicas são a porcentagem de sub-rotinas de programa e a porcentagem de declarações de programa chamadas durante a execução do conjunto de testes.

\subsection{Teste Multiobjetivo Baseado em Busca}

A princípio, os casos de testes era individuais, ou seja, as funções de adaptação era construídas com base no caminho em que seria percorrido. Ultimamente, se propõe testes com multiobjetivos. Isso principalmente pela crescente estudo de aspectos não-funcionais nos testes.









