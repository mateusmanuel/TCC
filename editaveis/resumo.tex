\begin{resumo}
 
Geração de dados de teste é um desafio corrente em Engenharia de Software e desenvolver um conjunto exato de dados para os testes nem sempre é uma tarefa simples de ser realizada.  Uma estratégia que vem sendo empregada é tratar a geração de dados de teste como um problema de otimização e com isso aplicar métodos de aproximação, também chamados de meta-heurísticas. Um problema identificado nesse processo de geração de dados é que os testes têm sido encarados como tendo objetivos únicos, geralmente analisando apenas a cobertura de código que está sendo alcançada. Contudo, figura não ser um cenário realista haja visto que outros objetivos podem (e devem) ser considerados ao propor um conjunto de testes. Como exemplo, aparenta existir um equilíbrio entre os esforços computacionais para a geração dos dados de teste e a execução da sua cobertura.  Além disso, ainda são necessários estudos que visem a análise de desempenho das meta-heurísticas empregadas na geração / execução de testes. Desse modo, o objetivo desse trabalho é investigar e analisar as relações a) entre a cobertura de código e o esforço computacional empregado na geração de dados de teste e b) a cobertura e esforço computacional na execução dos testes. Para tal objetivo, foi analisado o comportamento gráfico do algoritmo genético, meta-heurística bastante empregada na geração de dados de teste, sob o olhar das finalidades apresentadas. Como consequência, determinou-se o esforço que se destaca e realizou-se uma investigação em especial a fim de estabelecer parâmetros que possam incrementar esse tipo de meta-heurística para  geração de dados de teste. 
 \vspace{\onelineskip}
    
 \noindent
 \textbf{Palavras-chaves}: Problema de Otimização, Algoritmo Genético, Geração de Dados de Teste, Cobertura de Código, Esforço Computacional.
\end{resumo}
