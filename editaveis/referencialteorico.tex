% \part{Referencial Teórico}

\chapter[Referencial Teórico]{Referencial Teórico}

A busca pela "melhor" configuração ou conjunto de parâmetros para um determinado objetivo é uma preocupação recorrente de problemas práticos e teóricos \cite{combinatorialoptimization1998}. Em muitos desses problemas, realizar uma busca exaustiva por todos os estados a fim de encontrar uma condição não é viável devido ao tamanho de possibilidades a serem testadas. 

Para a solução desses problemas, métodos de aproximação (heurísticas) que geram resultados de alta qualidade, em um tempo razoável para uso prático, vem sendo adotados \cite{gendreau2005metaheuristics}. Com o desenvolvimento dessa área de pesquisa, esses modos de aproximação foram sendo melhor estruturados e conceituados, criando um nível generalista de heurísticas (meta-heurísticas).

% Essa conceituação abaixo é necessária? Ou com o final do paragrafo anterior é possível entender o que são meta-heurísticas? Seria interessante colocar algum exemplo? Ou até mesmo falar quando foi primeiro utilizado esse termo de heurísticas na solução de problemas de otimização?
Os métodos de busca meta-heurísticos podem ser definidos como metodologias, modelos, de nível superior que podem ser usados como estratégias de orientação no projeto de heurísticas subjacentes para resolver problemas específicos de otimização \cite{talbi2009metaheuristics}.

% É importante voltar tanto assim? Ou quebra muito o contexto? Vi que quebrou um pouco o ritmo de leitura, mas achei interessante mencionar
A ciência da computação, desde seus primórdios, também lida comumente com problemas de otimização, como é mostrado por Menabrea e Lovelace em suas notas - especialmente a nota D, em que relatam que é preferível escolher um arranjo que tenderá a reduzir o tempo necessário para completar o cálculo, se referindo a máquina de calcular de Babbage, do que necessariamente o arranjo exato \cite{menabrea1842sketch}.

Na comunidade acadêmica, as linhas de pesquisa que empregam meta-heurísticas para a resolução de problemas de otimização são denominadas baseadas em busca, do inglês \textit{search-based}.

\section{Teste de software baseado em busca}

Na engenharia de software, o primeiro trabalho com o uso de técnicas para otimização foi na área de testes. Em 1969, King utilizou a execução simbólica automatizada para checar a consistência entre o programa e suas proposições \cite{king1969program}. 

Anos depois surgiram os primeiros trabalhos utilizando meta-heurísticas. Como é o caso do SELECT, sistema para teste e depuração de programas utilizando execução simbólica aliada com a meta-heurística de subida da encosta \cite{boyer1975select}.

% É necessário explicar melhor o que esses trabalhos contribuíram?
Na mesma época, a utilização de meta-heurísticas se estendeu para a geração de caso de teste, área que ganhou enorme apreço por pesquisa. O trabalho de geração automática de dados de ponto flutuante \cite{miller1976automatic} acabou sendo um marco para os testes de software. Bem como, anos depois, a utilização de algoritmo genético para geração de dados de teste para a cobertura estrutural \cite{xanthakis1992application}.

A partir de então, as pesquisas foram combinando entre as meta-heurísticas - subida de encosto, algoritmos genéticos, colônia de formigas, recozimento simulado - e os diversos tipos de testes  conhecidos - testes estruturais, testes funcionais e não funcionais, testes de segurança, testes de robustez, testes de estresse, testes de mutação, testes de integração e testes de exceção \cite{harman2009search}.

Além da área de teste, outros ramos da engenharia de software começaram a utilizar as técnicas de busca, como foi o caso da estimativa de custos e do gerenciamento de projetos. O surgimentos de trabalhos fez a necessidade de melhor estruturar e definir a engenharia de software baseada em busca.

\subsection{Engenharia de software baseada em busca}

O termo engenharia de software baseada em busca foi pela primeira vez expressado por o manifesto \cite{harman2001search}, que propôs a reformulação da engenharia de software como um problema de busca. 


Mostrar relevância de SBSE para ESW

\subsection{Problemas pesquisados}

Para cada problema, caracterizá-lo:

- Definição breve e clara do problema

- Propostas de solução(fw, algoritmo, arquitetura...)

- Métodos de otimização empregrado(Alg. Genet., A*, Hill Climbing)

- Resultados alcançados

\subsection{Algoritmos mais utilizados}







